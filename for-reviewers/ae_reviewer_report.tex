\documentclass[conference]{IEEEtran}


\usepackage{cite}
\usepackage{amsmath,amssymb,amsfonts}
\usepackage{algorithmic}
\usepackage{graphicx}
\usepackage{textcomp}
\usepackage{xcolor}
\usepackage{booktabs}
\usepackage{ae_review}

\begin{document}

\title{Artifact Evaluation Report}

\author{\IEEEauthorblockN{Given Name Surname}
\IEEEauthorblockA{\textit{dept. name of organization} \\
\textit{name of organization}\\
City, Country \\
email address or ORCID}
\and
\IEEEauthorblockN{Given Name Surname}
\IEEEauthorblockA{\textit{dept. name of organization} \\
\textit{name of organization}\\
City, Country \\
email address or ORCID}
}

\maketitle

\aeroverview

\begin{center}
\begin{tabular}{cccc}
\toprule
Artifact ID & Available & Functional & Replicated \\
\midrule
$A_1$ &  \aerstatus{1} & \aerstatus{1} & \aerstatus{1} \\
$A_2$ &  \aerstatus{1} & \aerstatus{1} & \aerstatus{1} \\
$A_3$ &  \aerstatus{1} & \aerstatus{0} & \aerstatus{0} \\
$\ldots$ &  \\
\midrule
Badge awarded &  yes & no & no \\
\bottomrule
\end{tabular}
\end{center}

\aerrepoduction

\aerwhen

\begin{aerhint}
\begin{itemize}
    \item State when the experiments were done.
    \item e.g., "The artifact evaluation was conducted from July 24, 2025, to July 31, 2025."
    \item e.g., "Artifact evaluation was carried out on July 24, 2025."
\end{itemize}
\end{aerhint}

\aerwhere

\begin{aerhint}
\begin{itemize}
    \item Specify the computational resources utilized for experiments.
    \item e.g., "The experiments conducted for artifact evaluation were performed on Chameleon Cloud."
    \item Provide more details if different computational resources were used for analyzing different computational artifacts.
\end{itemize}
\end{aerhint}

\aerwhat


\begin{aerhint}
\begin{itemize}
    \item Describe the procedures undertaken to reproduce the computational artifacts detailed in the AD/AE Appendices.
    \item The detail level in this section can differ. If the procedures work as outlined in the AD/AE Appendices, a reference to these appendices suffices, confirming successful execution of the steps.
    In such instances, a brief statement for each computational artifact is sufficient.
    \item Generally, writing a brief paragraph about either the successful reproduction or the absence thereof of each computational artifact created by the authors is preferable.
    \item When facing issues in reproducing the computational artifacts, offer a high-level overview of the encountered problems. Omitting specific details is acceptable. For instance, it is sufficient to state that code compilation failed on machine X with compiler Y, without including the exact error messages.
    \item In the AE Report, potential shortcomings in the provided AD/AE Appendices that should be reported could include:
    \begin{itemize}
        \item \textbf{Unanticipated environmental variables:} Differences in computational environments that were not accounted for in the AD/AE Appendices.
        \item \textbf{Implicit assumptions:} Assumptions made by the original authors that are not explicitly documented.
        \item \textbf{Versioning issues:} Discrepancies arising from different versions of software or dependencies that were not detailed in the appendices.
        \item \textbf{Undocumented steps or procedures:} Any critical steps or procedures that the authors may have inadvertently left out of the appendix.
        \item \textbf{Data accessibility:} Challenges in accessing or using the data required for reproduction, which might not be covered in the appendices.
        \item \textbf{User-defined parameters:} Lack of clarity about user-defined parameters or settings required for running the experiments.
        \item \textbf{Hardware-specific issues:} Certain computational artifacts may behave differently on varying hardware configurations, which might not be addressed in the appendices.
        \item \textbf{Network dependencies:} Any network-related dependencies or configurations that were not explicitly mentioned.
    \end{itemize}
\end{itemize}
\end{aerhint}

\aerdisclaimer

\end{document}
